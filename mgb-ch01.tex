%mgb-ch01
%Chapter I - MGB Solutions
\begin{enumerate}
	%1-1
	\item[1.] \begin{enumerate}
		
		\item $\Omega = \{(B,W), (B,G), (G,W), (G,G)\}$.  The sample space contains four outcomes; an outcome itself is a 2-tuple where the first component represents the result of drawing from urn one and the second component from urn two.
		
		\item The even space is the collection of all subsets of the sample space. \\
		There are 16 such subsets. \\
		${\mathscr A} = \Big\{ \phi, \Omega, \{(B,W)\}, \{(B,G)\}, \{(G,W)\}, \{(G,G)\}$, \\
		$\phantom{{\mathscr A} = }\{(B,W), (B,G)\}, \{(B,W),(G,W)\}, \{(B,W), (G,G)\}, \{(B,G), (G,W)\}$, \\ 
		$\phantom{{\mathscr A} = }\{(B,G),(G,G)\}, \{(G,W), (G,G)\}, \{(B,W), (B,G), (G,W)\}$,\\ 
		$\phantom{{\mathscr A} = }\{(B,W), (B,G), (G,G)\}, \{(B,W), (G,W), (G,G)\}, \{(B,G), (G,W), (G,G)\} \Big\}$

		\item 1/4
		
		\item 0
	
		\end{enumerate}
	%1-2
	\item[2.] \begin{enumerate}
		
		\item There are many ways to describe the outcomes of this experiment.  For example, one could number the balls in urn one as 1, 2, 3 red; 4, 5 white and 6 blue and those in urn two as 1 red, 2, 3 white; and 4, 5, 6 blue.
		
		\begin{enumerate}
			
			\item Then $\Omega = \{(i_1,i_2)\colon i_1=1,\ldots,6$ and $i_2=1,\ldots, 6$, where \\
			$i_1$ is the number on the ball drawn from urn 1 and \\
			$i_2$ is the number on the ball drawn from urn 2.$\}$\\
			Note that there are 36 outcomes in this experiment. 
		
			\item \begin{enumerate}
				\item[Let] $A$ denote the event both balls are red. 
				\item[] $B$ denote the event both balls are white, and 
				\item[] $C$ denote the vent both balls are blue. 
			\end{enumerate}
				Then $P[\text{both balls are same color}] = P[A\cup B\cup C]$ $= P[A]+P[B]+P[C] =  \dfrac{3}{36} + \dfrac{4}{36} + \dfrac{3}{36}$.
			
			\item $ P[A] = \dfrac{3}{36} < \dfrac{4}{36} = P[B]$
		\end{enumerate} 
		
		\item $\text{(i)}\;\;\; \dfrac{12\cdot 8\cdot 4}{12^3}\quad\quad$ $\text{(ii)}\;\;\; \dfrac{12\cdot 8\cdot 4}{12\cdot 11\cdot 10}$
		
	\end{enumerate}
	
	\newpage
	%1-4
	\item[4.] \begin{enumerate}
	
		\item $\Omega = \{(i_1,i_2)\colon i_1=1,\ldots,5$ and $i_2=1,\ldots, 5$, where 
	$i_1$ is the number on the first ball drawn and 
	$i_2$ is the number on the second ball drawn$\}$. \\
	$B_1 = \{(i_1,i_2)\colon i_1=1,2,3$ and $i_2=1,\ldots, 5 \}$\\
	$B_2 = \{(i_1,i_2)\colon i_1=1,\ldots,5$ and $i_2=1, 2, 3 \}$\\
	$B_1B_2 = \{(i_1,i_2)\colon i_1=1,2,3$ and $i_2=1, 2, 3 \}$
	
		\item $P[B_1] = \dfrac{\text{size of}\ B_1}{\text{size of}\ \Omega} = \dfrac{3\cdot 5}{5\cdot 5}$.
		
		\item  $\Omega = \{(i_1,i_2)\colon i_1=1,\ldots,5$ and $i_2=1,\ldots, 5$ but $i_1\ne i_2\}$. \\
		$B_1 = \{(i_1,i_2)\colon i_1=1,2,3$ and $i_2=1,\ldots, 5$ but $i_1\ne i_2 \}$. \\
		$P[B_1] = \dfrac{3\cdot 4}{5\cdot 4}$, etc.
		\end{enumerate} 
	
	%1-7
	\item[7.] Using H for hit, M for miss, R for right hand and L for left hand, the event the the participant is successful is \\
	$\{(H,H,H), (H,H,M), (M,H,H)\} = A$, say. \\
	Under strategy RLR, $P[A] = p_1p_2p_1 + p_1p_2(1-p_1) + (1-p_1)p_2p_1$ and \\
	under strategy LRL, $P[A] = p_2p_1p_2 + p_2p_1(1-p_2) + (1-p_2)p_1p_2$. 
	 
	%1-8 
	\item[8.] \begin{enumerate}
		\item[(b)] $\dsp P[A\ \text{will beat}\ B\ \text{in three out of four}] = p^3+3p^3(1-p) = {4\choose 3}p^3(1-p) + p^4$ \\
		$\dsp P[A\ \text{will beat}\ B\ \text{in five out of seven}] = p^5 + 5p^5(1-p) + 15p^5(1-p)^2$ \\
		$\dsp \phantom{P[A\ \text{will beat}\ B\ \text{in five out of seven}]} = {7\choose 5}p^5(1-p)^2 + {7\choose 6}p^6(1-p) + p^7$
	\end{enumerate}

	%1-10
	\item[10.] $A = B$ and $p=1/2$ is a counterexample.
	
	%1-14
	\item[14.] $P[AB] = P[A]+P[B]-P[A\cup B] \ge P[A] +P[B]-1  = 1-\alpha - \beta$.
	
	%1-18
	\item[18.] \begin{enumerate}
		\item[(a)] $(1/3)^4$
		\item[(b)] $3(1/3)^4$
		\item[(c)] $3(1/3)^4 + 4\cdot 3(1/3)^4 = 5/27$
	\end{enumerate}

	%1-19
	\item[19.]\begin{enumerate}
		\item[(a)] $P[\text{total of}\ 9] = 25/216$; $P[\text{total of}\ 10] = 27/216$
		\item[(b)] $P[\text{at least one}\ 6\ \text{in }\ 4\ \text{tosses}] = 1- (5/6)^4$\\
				$P[\text{at least double}\ 6\ \text{in }\ 24\ \text{tosses}] = 1- (35/36)^{24}$
		\item[(c)] $P[\text{at least one}\ 6\ \text{with }\ 6\ \text{dice}] = 1- (5/6)^6$\\
		$P[\text{at least two}\ 6\text{'s with }\ 12\ \text{dice}] = 1- (5/6)^{12} - (12)(1/6)(5/6)^{11}$
	\end{enumerate}

	\newpage
	%1-20
	\item[20.] This is similar to Problem 27.
	
	%1-22
	\item[22.] $(365)_{25}/(365)^{25}$.
	
	%1-23
	\item[23.] $\dfrac{\dsp {5\choose 2}{21\choose 11}}{\dsp {26\choose 13}} + \dfrac{\dsp {5\choose 3}{21\choose 10}}{\dsp {26\choose 13}}$.
	
	%1-24
	\item[24.] (a) $\dfrac{\dsp {r\choose k}(n-1)^{r-k}}{n^r}$

	%1-25
	\item[25.] Consider that a single coin is tossed until the first head occurs. \\
				$P[\text{first head occurs on toss}\ j] = (1/2)^j$ \\
				$P[\text{Ace wins}] = (1/2) + (1/2)^5 + (1/2)^7 + \ldots = 4/7$. \\
				$P[\text{Bones wins}] = (1/2)^2 + (1/2)^5 + (1/2)^8 + \ldots = 2/7$. \\
				$P[\text{Clod wins}] = (1/2)^3 + (1/2)^6 + (1/2)^9 + \ldots = 1/7$. 
	
	%1-26
	\item[26.] $P[\text{single ring formed}] = (4/5)(2/3)$. \\
				$P[\text{at least one ring formed}] = 1$. 
		
	%1-27		
	\item[27.] You might test your intuition on this one and guess the answer before you proceed.  Let $A_1 = \{\text{Mr. Bandit does not get caught under strategy 1}\}$ where strategy 1 is to sell all twenty at once; strategy 2 is to put four stolen cattle in one set of ten; strategy 3 is to put three stolen cattle in one set of ten and one on the other; and strategy 4 is to put two stolen cattle in each set of ten.
	\begin{eqnarray*}
		P[A_1] &=& \dfrac{\dsp {4\choose 0}{16\choose 4}}{\dsp {20\choose 4}} \\
		P[A_2] &=& \dfrac{\dsp {4\choose 0}{6\choose 2}}{\dsp {10\choose 2}} \cdot \dfrac{\dsp {10\choose 2}}{\dsp {10\choose 2}} \\
		P[A_3] &=& \dfrac{\dsp {3\choose 0}{7\choose 1}}{\dsp {10\choose 2}} \cdot \dfrac{\dsp {1\choose 0}{9\choose 1}}{\dsp {10\choose 2}} \\
		P[A_4] &=& \dfrac{\dsp {2\choose 0}{8\choose 2}}{\dsp {10\choose 2}} \cdot \dfrac{\dsp {2\choose 0}{8\choose 2}}{\dsp {10\choose 2}}				
	\end{eqnarray*}

	\newpage
	%1-30
	\item[30.] (b) Take $B=C$ and $P[A]>P[B]$ for a counterexample.
	
	%1-31
	\item[31.] Use the corollary of Thorem 29.
	
	%1-32
	\item[32.] This is similar to Problem 70.  Use Bayes' Formula.
	
	%1-33
	\item[33.] (c) $\begin{array}{|cccc|}
	\multicolumn{1}{c}{\cdot} & \cdot & \cdot &\multicolumn{1}{c}{\cdot} \\ 
	\multicolumn{1}{c}{\cdot} & \cdot & \cdot & \multicolumn{1}{c}{\cdot} \\ \hline
	\cdot & \cdot & \cdot & \cdot \\ 
	\cdot & \cdot & \cdot & \cdot \\ \hline 
	\multicolumn{4}{c}{A}
	\end{array}$ \;
	$\begin{array}{|cc|cc} \cline{1-2}
	\cdot & \cdot & \cdot & \cdot \\ 
	\cdot & \cdot & \cdot & \cdot \\ 
	\cdot & \cdot & \cdot & \cdot \\ 
	\cdot & \cdot & \cdot & \cdot \\ \cline{1-2} 
	\multicolumn{4}{c}{B}
	\end{array}$ \;
	$\begin{array}{|cccc|} 
	\multicolumn{1}{c}{\cdot} & \cdot & \cdot &\multicolumn{1}{c}{\cdot} \\ \hline 
	\cdot & \cdot & \cdot & \cdot \\ \hline 
	\multicolumn{1}{c}{\cdot} & \cdot & \cdot & \multicolumn{1}{c}{\cdot} \\ \hline 
	\cdot & \cdot & \cdot & \cdot \\ \hline
	\multicolumn{4}{c}{C}
	\end{array}$ \;
	$\begin{array}{|c|c|c|c} \cline{1-1}\cline{3-3}
	\cdot & \cdot & \cdot & \cdot \\ 
	\cdot & \cdot & \cdot & \cdot \\ 
	\cdot & \cdot & \cdot & \cdot \\ 
	\cdot & \cdot & \cdot & \cdot \\ \cline{1-1}\cline{3-3}
	\multicolumn{4}{c}{D}
	\end{array}$ \;
	
	%1-34
	\item[34.] Use Theorem 29.
	
	%1-39
	\item[39.] Use Bayes' Formula.
	
	%1-40
	\item[40.] This problem is known as the "liars problem." It can be varied by changing the number of liars.  In fact, the reader might want to try to solve it for only two or three liars before reading the solution.  As is the case with most "story" problems some "modelling" is required.  Let $A_T = \{$statement that A makes is true$\}$, and $D_T = \{$D says that C says that B says that A is telling the truth$\}$, then $P[A_T\vert D_T]$ is what is sought.  Also, let \\
	$B_T = \{$B says that A is telling the truth$\}$, and \\ 
	$C_T = \{$C says that B says that A is telling the truth$\}$. \\
	Note that $C_T = \{$C says that B says that A is not telling the truth$\}$ and similarly for $\overline{B}_T$ and $\overline{D}_T$.  Actually some modelling" has been done in defining these events; for example, it has been assumed that B does say that A's statement is either true or false.  Note that \\
	$1/3 = P[A_T] = P[B_T\vert A_T] = P[C_T\vert B_T] +P[D_T\vert C_T] = P[C_T\vert B_TA_T] = P[D_T\vert C_TA_T]$, and \\
	$2/3 = P[\overline{A}_T] = P[B_T\vert \overline{A}_T] = P[C_T\vert \overline{B}_T] = P[D_T\vert \overline{C}_T] = P[C_T\vert \overline{B}_TA_T] = P[D_T\vert \overline{C}_TA_T]$. \\
	Implicitly, it has been assumed that not only does each liar lie with probability 2/3 in any given instance, but also the liars lie independently of each other.  The solution given here includes the solution for the two and three liars problems. \\
	$P[B_T] = P[B_T\vert A_T]P[A_T] + P[B_T\vert \overline{A}_T]P[\overline{A}_T] = (1/3)(1/3) + (2/3)(2/3) = 5/9$, so \\
	$P[A_T\vert B_T] = \dfrac{P[B_T\vert A_T]P[A_T]}{P[B_T]} = 1/5$, the solution to the two liar problem.
	\begin{eqnarray*}
	\text{Now}\ P[C_T] &=& (1/3)(5/9) + (3/2)(4/9) \;=\; 13/27\ \text{and} \\
	    P[C_T\vert A_T] &=& P[C_TB_T\vert A_T] + P[C_T\overline{B}_T\vert A_T] \\
	    	&=& P[C_T\vert B_TA_T]P[B_T\vert A_T] + P[C_T\vert \overline{B}_TA_T]P[\overline{B}_T\vert A_T] \\
	    	&=& 5/9,\ \text{hence} \\
	    P[A_T\vert C_T] &=& \dfrac{(5/9)(1/3)}{(13/27)},\ \text{the solution to the three liar problem.}	
	\end{eqnarray*} 
	\begin{eqnarray*}
	\text{Similarly,}\ P[D_T] &=& P[D_T\vert C_T]P[C_T] + P[D_T\vert \overline{C}_T]P[\overline{C}_T] \\
	&=& (1/3)(13/27) + (2/3)(14/27) \;=\; 41/81,\ \text{and} \\
	\end{eqnarray*}
	\begin{eqnarray*}
	P[D_T\vert A_T] &=& P[D_T\vert C_TA_T]P[C_T\vert A_T] + P[D_T\vert \overline{C}_TA_T]P[\overline{C}_T\vert A_T] \\
	&=& (1/3)(5/9) + (2/3)(4/9) \;=\; 13/27,\ \text{and} \\
	\text{finally},\hfill && \\
	P[A_T\vert D_T] &=& \dfrac{(13/27)(1/3)}{41/81} \;=\; \dfrac{13}{41}.
	\end{eqnarray*}

	%1-42
	\item[42.] \begin{enumerate}
		\item[(a)] 2/3 
		\item[(b)] 4/5
		\item[(c)] 1
	\end{enumerate}	
	
	%1-46
	\item[46.] A and B disjoint and $P[A] \ne P[B]$ gives a counter-example.
	
	%1-48
	\item[48.] Let $A_j = \{$exactly $j$ seeds out of the fifty germinate$\}$. \\
	Model by assuming each seed germinates with probability 0.95.  P[package will violate guarantee] = \\
	$\dsp \sum_{j=0}^{44} P[A_j] = 1 - \sum_{j=45}^{50}P[A_j] = 1 - \sum_{j=45}^{50}(.96)^j(.04)^{50-j}$.
	
	%1-50
	\item[50.] Intuition says the answer ought to be greater than 1/2. \\
	Let A = \{tested stone is real\} \\
	\phantom{Let} B = \{son gets real diamonds\}
	We want $P[B\vert A]$ and $P[B\vert \overline{A}]$.  Symmetry suggests that these two conditional probabilities are equal. \\
	Define C = $\{$box with two real diamonds is selected for testing$\}$ and model by assuming $P[C]=1/2$, $P[A\vert C]=2/3$, and $P[A\vert \overline{C}]=1/3$.
	Then $P[A]=P[A\vert C]P[C] + P[A\vert \overline{C}]P[\overline{C}] =2/3\cdot 1/2+1/3\cdot 1/2 = 1/2$. 
	\begin{eqnarray*}
	P[B\vert A] &=& \dfrac{P[AB]}{P[A]} \;=\; \dfrac{P[AB\vert C]P[C] + P[AB\vert \overline{C}]P[\overline{C}]}{P[A]} \\
	&=& \dfrac{(2/3)(1/2)+(0)(1/2)}{1/2} \;=\; 2/3.\ \text{Similarly} \\
	P[B\vert \overline{A}] &=& \dfrac{(0)(1/2)+(2/3)(1/2)}{1/2} \;=\; 2/3.
	\end{eqnarray*}
	
	%1-57
	\item[57.] Let A = $\{$player wins$\}$.  Let $B_j = \{$total of $j$ on first toss$\}$. \\
	$\dsp P[A] = \sum_{j=2}^{12}P[A\vert B_j]P[B_j]$.
	
	%1-59
	\item[59.] \begin{enumerate}
		\item[(a)] $p^4 + 4p^3(1-p) + 4p^2(1-p)^2 = a$ (say)
		\item[(b)] $p^4 +4p^3(1-p) +2p^2(1-p)^2 =b$ (say)
		\item[(c)] $pa + (1-p)b$		
	\end{enumerate}

	%1-62
	\item[62.] Mark first in a corner.  The random player must then mark in the center to keep you from winning.  Next mark one of the two spaces adjacent to your first mark, etc.  Your opponent's chance of forcing a tie under this strategy is (1/8)(1/6)(1/4)(2/2).  No other strategy does better.  Your chance of winning is 191/192.  How does the problem change if you allow your opponent to mark first?
	
	%1-63
	\item[63.] Apply Bayes' Formula.
	
	%1-67
	\item[67.] 3/4; 1/3

	%1-68
	\item[68.] \begin{enumerate}
		\item[(a)] Outcomes are yellow-smooth (Y-S), yellow-wrinkled (Y-W), green-smooth (G-S), and green-wrinkled (G-W); they are equally likely.
		\item[(b)] $\begin{array}{c|c|c|c}
			Y-S & Y-W & G-S & G-W \\ \hline
			3/8 & 1/8 & 3/8 & 1/8 
		\end{array}$
		\item[(c)] $\begin{array}{c|c|c|c}
			Y-S & Y-W & G-S & G-W \\ \hline
			9/16 & 3/16 & 3/16 & 1/16 
		\end{array}$
	\end{enumerate}

	%1-70
	\item[70.] \begin{enumerate}
		\item[(a)] $\dsp P[B\vert A] = \dfrac{P[A\vert B]P[B]}{P[A\vert B]P[B] +P[A\vert \overline{B}]P[\overline{B}]} = \dfrac{(.95)(.05)}{(.95)(.05) + (.05)(.95)} = \dfrac{1}{2}$.
		\item[(b)] $.9 = \dfrac{p(.05)}{p(.05) + (1-p)(.95)}$ implies $p=\dfrac{17.1}{17.2} = .9942$.
	\end{enumerate}

\end{enumerate}







