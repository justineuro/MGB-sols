%mgb-ch10
%Chapter X - MGB Solutions
Problems 1 through 6 are solved by using the given data and appropriate formulas in Sections 4 and 5.
\begin{enumerate}
	\item[1.] Equations 7, 8, and 9.
	
	\item[2.] See the Corollaries of Theorem 2.
	
	\item[3.] Equations 15, 16, and 14.
	
	\item[4.] See Page 494.
	
	\item[5.] Use the invariance property; see Theorem 2 on page 285.
	
	\item[6.] Similar to Problem 8 below.
	
	\item[7.] $P[Y_{x_0}\le \beta_0 + \beta_1x_0 + z_p\sigma] = \Phi\left(\dfrac{\beta_0+\beta_1x_0+z_p\sigma -(\beta_0+\beta_1x_0)}{\sigma}\right) = \Phi(z_p) = p$.
	
	\item[8.] $\hat{\beta}_0 + \hat{\beta}_1x + z_p[\Gamma((n-2)/2)/\sqrt{2}\Gamma((n-1)/2)][\sum\limits_{1}^{n}(y_i-\hat{\beta}_0-\hat{\beta}_1x_i)^2]^{1/2}$.
	
	\item[10.] $\hat{\beta}_0\approx .497, \hat{\beta}_1\approx 2.049, \hat{\sigma}^2\approx .00117,$ and $\hat{\mbox{var}}[\hat{\beta}_1]\approx .00255.$ A 95\% confidence interval estimate for $\beta_1$ is (1.93, 2.17). $\beta_1=1$ is outside this interval, so according to the confidence interval technique, the hypothesis $\beta_1=1$ may be rejected.
	
	\item[11.] Similar to Problem 10. 

	\item[12.] Could set a one-sided confidence interval on $\mu(.50)$ and use the confidence interval technique.
	
	\item[13.] Use the invariance property of confidence intervals. See the Remark on Page 378.

	\item[14.] $\hat{\beta}_1(n-4)/\sum(y_i-\hat{\beta}_0-\hat{\beta}_1x_i)^2$

	\item[15.] $\hat{\beta}_1 = \dfrac{(\sum a_i)(\sum a_ix_iy_i) -(\sum a_iy_i)(\sum a_ix_i)}{(\sum a_i)(\sum a_ix_i^2) - (\sum a_ix_i)^2}$, \\
	$\hat{\beta}_0 = (\sum a_iy_i - \hat{\beta}_1\sum a_ix_i)/\sum a_i$, and \\
	$\hat{\sigma}^2 = \sum a_i(y_i-\hat{\beta}_0-\hat{\beta}_1x_i)^2/n$.

	\newpage

	\item[16.] Recall that $\hat{B}_0$ and $\hat{B}_1$ have a bivariate normal distribution. What is required for independence in a bivariate normal?
	
	\item[17.] $\mbox{cov}[\overline{Y},\hat{B}_1] = \mbox{cov}[\hat{B}_0+\hat{B}_1\overline{x}, \hat{B}_1] = \mbox{cov}[\hat{B}_0, \hat{B}_1] + \overline{x}\ \mbox{var}[\hat{B}_1] = 0$ by Equation (12). \\
	$\overline{Y}$ and $\hat{B}_1$ have a bivariate normal distribution so uncorrelated implies independence. 
	
\end{enumerate}

\noindent Problems 19, 20, and 21 can be worked using the theory of Lagrange multipliers as in the proof of Theorem 6.
