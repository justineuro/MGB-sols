%mgb-ch03
%Chapter III - MGB Solutions
\begin{enumerate}
	%3-1
	\item[1.] \begin{enumerate}
		
		\item[(f)] No, the variance of a negative binomial random variable cannot be smaller than its mean.
		
		\item[(h)] Rectangular, normal, logistic, and beta with $a=b$.  Note that the binomial for $p=1/2$ and $n$ even does not work.
		
		\item[(n)] No.
		
		\item[(o)] Yes, if the distribution of $X$ is symmetric about zero.
		
	\end{enumerate}
	%1-2
	\item[2.] \begin{enumerate}
		
		\item[(b)] If $r\le 1$, the mode is zero.  If $r>1$, the mode is $(r-1)/\lambda$.
	\end{enumerate}
	%1-4
	\item[4.] \begin{enumerate}
		
		\item[(b)] $2\Phi(-2)$
		
		\item[(c)] $P[X\le 0] = \Phi(-\mu/\sqrt{h(\mu)}) = \Phi(-1/\sqrt{a})$ for $h(\mu)= a\mu^2,\ \mu >0$.
	\end{enumerate} 
	
	\item[6.] Let $X$ be a random variable denoting the low bid of the competition.  $X$ is uniformly distributed over the interval $((3/4)C,2C)$.  Let $P$ denote profit and $B$ the amount the contractor should bid.  Now $P=(B-C)I_{(B,2C)}(X)$ and 
	\begin{eqnarray*}
	E[P] &=& \int (B-C)I_{(B,2C)}(x)f_X(x)\ dx \;=\; (B-C)\int_{(3/4)C}^{2C} I_{(B,2C)}(x)\left(2C-\dfrac{3}{4}\right)^{-1}\ dx \\
	&=& \dfrac{(B-C)}{\left(\dfrac{5}{4}\right)C}(2C-B).\ \text{Now maximize with respect to}\ B\ \text{and obtain}\ B =\dfrac{3C}{2}.
	\end{eqnarray*} 
	
	\item[7.] \begin{enumerate}
		\item[(a)] Let $k =$ number he should stock and $X$ the number he can sell in 25 days. \\
		Want the minimal $k$ such that $P[X\le k] \ge .95$ where $X$ has a Poisson distribution with parameter $100$; that is, solve for $k$ in $\dsp \sum_{i=0}^{k} \dfrac{e^{-100}(100)^i}{i!} \ge .95$. \\
		From a table of the Poisson distribution, $k=117$ is obtained.  Using the normal approximation and $\dsp \Phi\left(\dfrac{k-100}{10}\right) = .95,\ k=117$ is obtained.

		\item[(b)] Let $Z =$ number of days out of 25 that he sells no items. \\
		Under appropriate assumptions (what are they?) $Z$ has a binomial distribution with $n=25$ and $p=c^{-4}$.  Hence, $E[Z] = 25c^{-4}$.
	
	\end{enumerate}
	
	\item[8.] \begin{enumerate}
		\item[(a)] $Y$ has a binomial distribution with parameters $n$ and $q$.
		
		\item[(b)] $X$ has a binomial distribution with parameters $n$ and $15/36$.
		
		\item[(c)] $(X+n)/2$ has a binomial distribution with parameters $n$ and $p$. \\
		Hence $E[X] = n(2p-1)$.
		
		\item[(d)] Show that $\dsp \sum_{j=0}^{k} {n\choose j}\left(p_1^jq_1^{n-j} - p_2^jq_2^{n-j}\right) = \sum_{j=0}^{k}d_j(\text{say}) \ge 0$. \\
		Note that $\dsp \sum_{j=0}^{n} d_j =0$, hence it suffices to show that the first few $d_j$'s are positive, and the remaining are negative.  But $d_j \ge 0$ if and only if \\ $j\le n\log(q_2/q_1)/log(p_1q_2/p_2q_1)$. \\
		(Use the result of Problem 28 for an alternate proof.)
		
	\end{enumerate}
	
	\item[9.] $\dsp \sum_{j=60}^{100} \dfrac{\dsp {2500 \choose j}{2500 \choose {100-j}}}{\dsp {5000\choose 100}}$.  The hypergeometric can be approximated by the binomial and the binomial can in turn be approximated by the normal which gives a numerical answer of approximately $1- \Phi(2) = .0228$.
		
	\item[11.] Let $X$ denote the number of defectives in the sample.  Assume that $X$ has a binomial distribution. \begin{enumerate}
		\item[(a)] $P[X\ge 1] = 1 - P[X=0] = 1 - (.99)^{10}$.
		\item[(b)] Want $P[X\ge 1]\approx .95;$ or, want $P[X=0] \approx .05$; \\
		i.e., $(.9)^n \approx .05$, or, $n\approx 29$.
	\end{enumerate}
	
	\item[15.] $\mu + c\left[\Phi\left(\dfrac{a-\mu}{\sigma}\right) - \Phi\left(\dfrac{b-\mu}{\sigma}\right)\right]/\left[\Phi\left(\dfrac{b-\mu}{\sigma}\right) - \Phi\left(\dfrac{a-\mu}{\sigma}\right)\right]$
	
	\item[17.] There is a misprint in this problem.  The mean was intended to be 200 rather than 20.  Want \\
	$P[X \ge 150] \ge .90$, or, $\Phi\left(\dfrac{50}{\sigma}\right) \ge .90$, which implies $\sigma \approx 50/1.282 \approx 39$.
	
	\item[19.] \begin{enumerate}
		\item[(a)] \begin{eqnarray*}
			E[X] &=& \int_{0}^{\infty} \beta^{-2}x^2\exp[-(1/2)(x/\beta)^2]\ dx \\
			&=& (1/2)\sqrt{2\pi}\beta^{-1}\int_{-\infty}^{\infty}x^2(1/\beta\sqrt{2\pi})\exp[-(1/2)(x/\beta)^2]\ dx \\
			&=& \beta\sqrt{2\pi}/2\ \text{by recognizing that the last integral is the variance of a} 
		\end{eqnarray*} 
		normal distribution with mean $0$ and variance $\beta^2,$ which shows how a little knowledge of probability can be an aid to integration.
		\begin{eqnarray*}
		var[X] &=& \beta^2(4-\pi)/2.
		\end{eqnarray*}
	
		\item[(b)] No.	
	\end{enumerate}

	\item[25.] $\begin{array}{c|c|c|c|c|c|c|c|c}
		1 & 2 & 3 & 4 & 5 & 6 & 7 & 8 & 9 \\ \hline
		\dfrac{9}{81} & \dfrac{12}{81} & \dfrac{16}{81}& \dfrac{12}{81}& \dfrac{12}{81} & \dfrac{10}{81}& \dfrac{6}{81}& \dfrac{3}{81}& \dfrac{1}{81}
	\end{array}$
	
	\item[28.] Assume true and differentiate both sides with respect to p to obtain the equality:
	\begin{eqnarray*}
	\sum_{j=k}^{n} j{n\choose j}p^{j-1}q^{n-j} - \sum_{j=k}^{n}(n-j){n\choose j}p^jq^{n-j-1} = k{n\choose k}p^{k-1}q^{n-k}.
	\end{eqnarray*} 
	The inequality is verified by noting the (j+1)st term of the first sum cancels the j\underline{th} term of the second sum.  Work backwards.
	
	\item[29.] Let $X$ = \# of successes in the first n Bernoulli trials \\
	and $Y$ = \# of failures prior to the rth success. \\
	Note that (X$\le$r-1)$\cong$(Y$>$n-r) hence $F_X(r-1)$ = $P[X\le r-1]$ = $P[Y> n-r]$  = $1 - F_Y(n-r)$.
	
	\item[30.]${}$\vspace{-7.5ex}\begin{eqnarray*}
	E[Z_\lambda] &=& (E[U^\lambda]-E[1-U^\lambda])/\lambda = 0\ \text{for}\ \lambda >-1. \\
	E[Z_\lambda^2] &=& (E[U^{2\lambda}] -2E[U^\lambda(1-U)^\lambda] + E[(1-U)^{2\lambda}])/\lambda^2 \\
	&=& (2/\lambda^2)([1/(2\lambda+1)] - B(\lambda+1,\lambda+1))\ \text{for}\ \lambda > -1/2. \\
	E[Z_\lambda^3] &=& 0\ \text{for}\ \lambda > -1/3. \\
	E[Z_\lambda^4] &=&(2/\lambda^4)([1/(4\lambda+1) - 4B(3\lambda+1,\lambda+1) + 3B(2\lambda+1,2\lambda+1)])\ \text{for}\ \lambda > -1/4.
	\end{eqnarray*}	
	The last part is misstated.  The intent was to get two different $\lambda$'s, say $\lambda_1$ and $\lambda_2$, such that $Z_{\lambda_1}$ and $Z_{\lambda_2}$ have the same skewness and kurtosis.  If $\lambda_1$ and $\lambda_2$ are sought so that $Z_{\lambda_1}$ and $Z_{\lambda_2}$ have kurtosis equal to zero, then $\lambda_1\approx .135$ and $\lambda_2\approx 5.20$ will work. 
	
\end{enumerate}

